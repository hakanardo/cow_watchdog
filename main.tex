\documentclass[10pt,a4paper,twocolumn]{article}
\usepackage[utf8]{inputenc}
\usepackage[english]{babel}
\usepackage{amsmath}
\usepackage{amsfonts}
\usepackage{amssymb}
\author{Håkan Ardö, Oleksiy Guzhva, Mikael Nilsson}
\title{Cow interaction watch dog}
\begin{document}
\maketitle

\begin{abstract}
Cows are interacting...
\end{abstract}

\section{Introduction}

Scientists working with animal behaviour and welfare are interested in studying the social interactions between cows in dairy farms. Typically that is performed by defining a set of interaction events such as head butting, body pushing, social licking etc. and writing very detailed protocol with description for every event. Then an expert studies an area of interest for large amount of time and counts the number of each behavioural event happening within pre-defined time-bouts. Some of these behavioural events are quite rare, which means that a lot of expert time have to be spent in looking at raw video data in order to find potentially interesting sequences.

In this paper the goal is to take the first step towards an automated system. The study area is filmed using video cameras. Then an automated watchdog system will remove a lot of the uninteresting parts of the recorded video material. The remaining video sequences will still have to be studied by experts, but the time spent looking at uninteresting video sequences will be significantly reduced.

The pilot study used to develop this watchdog was made in a dairy barn in the south of Sweden with 252 Swedish Holstein cows. They were milked by four automated milking robots, which had a common waiting area (6x18 meters). This waiting area is an common space which cows that are ready for milking could anter at any point in time. They will then interact with each other  in order to decide who are allowed to enter each of the milking robots first. These interactions, their number and ratio between positive and negative behavioural events are highly dependent on cow's rank in a herd hierarchy. Therefore, studies explaining the social component of animal behaviour are of great importance for animal welfare abd management. 

Video recordings were made using three Axis M3006-V cameras with a wide angle of 134 degrees and placed at 3.6 meter height pointing straight down to optimize overview over the study area. There is a significant overlap between the camera images in order to not miss events taking place at the border between the cameras. In total 2315 hours (1 month) of 800x600 video in 16 fps was collected.

The cameras was calibrated to compensate for lens distortion and rectified. Although the cameras was physically mounted to point fairly straight down, they was still slightly tilted. This tilting was synthetically removed during this rectification. The end result of this calibration is video images where the cows have the same size regardless of where in the image they appear. Also the scan-lines of the three different cameras becomes aligned which allows them to be stitched together to form an overview of the entire waiting area.

Finally a Convolution Neural Network (CNN)  was trained to detect the cows in the images, and statistics about how many cows and their distances/relation to each other was extracted. Using these statistics the XXX scientists can form queries to select time intervals of interest to watch, such as "show me video clips involving at least two cows with the neck of one cow closer than one meter to the body of the other".

\section{Brief technical about calibration apporach?}

\section{CNN cow detector}

A random subset of the full recording consisting of 1722 images was manually annotated. This subset contained in total 6399 cows. Each cow was annotated with seven landmark points: head, left and right shoulder, front middle, left and right hip and back middle. In addition to that one additional landmark "cow center" was defined as the mean of front middle and back middle. This data was then used to train a CNN detector.

The detector was split into two step. The first step is a fully convolutional CNN that detects the landmarks in the image. Currently only four of the landmarks was used to speed up the experiments, but extending to use all 8 is straight forward. The architecture of this network is a fully convolutional version of VGG \cite{Simonyan14c} with batch normalization \cite{DBLP:journals/corr/IoffeS15} added after each convolution step. Details are shown in Table \ref{tab:cownet}.

The second step is another CNN that works with the probability map produced by the first as input and tries to detect the cows and their orientations. The full circle is divided into 32 equally spaced orientations which generates 32 different oriented cow classes. In addition to that there is the no cow class, which makes the total number of classes of this CNN 33. The input probabilities turned into log likelihoods as it makes more sense when summing them together. Then the network consists of a single $ 13 \times 13 $ convolutional layer. Details are shown in Table \ref{tab:cowdirnet}.

\begin{table}
\begin{center}
\begin{tabular}{|l|c|c|}
\hline
\textbf{Layer type} & \textbf{Size} & \textbf{Channels} \\
\hline

Conv + BNorm + Relu & 3x3 & 32 \\
MaxPool(stride=2) & 2x2 &  \\
\hline

Conv + BNorm + Relu & 3x3 & 64 \\
MaxPool(stride=2) & 2x2 &  \\
\hline

Conv + BNorm + Relu & 3x3 & 128 \\
Conv + BNorm + Relu & 3x3 & 128 \\
MaxPool(stride=2) & 2x2 &  \\
\hline

Conv + BNorm + Relu & 3x3 & 256 \\
Conv + BNorm + Relu & 3x3 & 256 \\
MaxPool(stride=2) & 2x2 &  \\
\hline

Conv + BNorm + Relu & 3x3 & 512 \\
Conv + BNorm + Relu & 3x3 & 512 \\
MaxPool(stride=2) & 2x2 &  \\
\hline

Conv + BNorm + Relu & 1x1 & 1024 \\
Conv + BNorm + Relu & 1x1 & 1024 \\
Conv + BNorm + Relu & 1x1 & 5 \\
Softmax & & \\
\hline

\end{tabular}
\end{center}
\caption{CNN architecture used to detect different landmarks of the cows. The input is an image of any size with 3 rgb channels scaled to the range $\left[0,\,1\right]$. The output is probability map segmenting the entire image into 5 classes: Ground, Cow front middle, Cow center, Cow back middle and Cow head.}
\label{tab:cownet}
\end{table}

\begin{table}
\begin{center}
\begin{tabular}{|l|c|c|}
\hline
\textbf{Layer type} & \textbf{Size} & \textbf{Channels} \\
\hline

MaxPool(stride=1) & 3x3 &  \\
Log & & \\
Conv + BNorm + Relu & 13x13 & 33 \\
Softmax & & \\
\hline
\end{tabular}
\end{center}
\caption{CNN architecture use to detect the cows and their orientation. The input is the 5 channel probability map from the landmark detector with the last MaxPool removed to increase resolution. The output is a probability map that segments the image into either background or cow in one of 32 different orientations.}
\label{tab:cowdirnet}
\end{table}

The landmark net were trained on patches of $150\times 150$ pixels extracted from the input images. This makes the output during training a single pixel. The positive examples were centred on the landmarks and randomly jittered $\pm 16$ pixels (as the distance between output pixels is $32$ input pixels). Negative patches where selected at centres more than $32$ pixels from any landmark. In addition to the positive and negative patches a set of don't care patches were selected at random centres at distances between $16$ and $32$ pixels from landmarks. The ground truth probability of these patches belong to the class of the landmark was set to $0.5$ and the probability that they are ground was set to $0.5$. In some cases several landmarks appears within $32$ pixels of the patch center. In that case the probability mass was distributed uniformly by all involved classes.

The weights of the convolutions are initiated using random samples draw from a Gaussian
distribution truncated at $2\sigma$, with standard deviation $\sigma=\sqrt{\frac{2}{n}}$,
where $n$ is the number of inputs\cite{DBLP:journals/corr/HeZR015}. The networks are regularized with weight decay of
$0.0001$ and optimized using stochastic gradient descent with $0.9$ momentum. The
learning rate is initiated to $1.0$ and reduced by a factor $10$ each time the validation
error flattens. The landmark CNN uses only valid outputs from the convolutional and maxpool
layers while the cow detector keeps the same resolution to also detect cows that are
slightly outside the image.

Once the net was trained, the last maxpool layer was removed to increase the output resolution. The net was then applied to the full rectified training images producing probability maps of $44\times 46\times 5$ pixels. These were used as training examples for the cow detection net (without splitting them into patches). Output ground truth probability maps of $44\times 46\times 33$ pixels were constructed from the annotations by projecting each cow, $i$, center point into the probability map as $\left( x_i, y_i \right)$ and calculate its angle $a_i$ as the angle of line between front middle and back middle. Then a binary $44\times 46\times 33$ mask $B\left( x, y, c \right)$ is formed, containing a background mask
\begin{equation}
B\left( x, y, 32 \right) = \left\lbrace
\begin{array}{clc}
0 & \text{if} &
\begin{array}{c}
 \lfloor x_i \rfloor \leq x \leq \lceil x_i \rceil \\
 \lfloor y_i \rfloor \leq y \leq \lceil y_i \rceil
\end{array}
\\
1 & \multicolumn{2}{l}{\text{otherwise}}
\end{array}
\right.
\end{equation}
and $32$ orientation masks
\begin{equation}
B\left( x, y, c \right) = \left\lbrace
\begin{array}{clc}
1 & \text{if} &
\begin{array}{c}
 \lfloor x_i \rfloor-1 \leq x \leq \lceil x_i \rceil+1 \\
 \lfloor y_i \rfloor-1 \leq y \leq \lceil y_i \rceil+1 \\
 \mathrm{adist}\left(\frac{2c\pi}{32}, c_i\right) < \frac{2\pi}{32} \\
\end{array}
\\
0 & \multicolumn{2}{l}{\text{otherwise}}
\end{array}
\right.
\end{equation}
for $0\leq c \leq 31$ and all $i$. The $\mathrm{adist}$ function calculates the absolute angular distance between two angles. The ground truth probability masks are then produced by normalising $B$ to sum to $1$ for each pixel. Finally, the network is trained using the same hyper parameters as described above.

XXX: Tracking?

\section{Watchdog}

\section{Results}

\section{Conclusions}

\bibliographystyle{plain}
\bibliography{main}

\end{document} 
